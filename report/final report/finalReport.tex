\documentclass[11pt,a4paper]{scrreprt}
\usepackage[utf8]{inputenc}
\usepackage[english]{babel}
\usepackage{amsmath}
\usepackage{amsfonts}
\usepackage{titling}
\usepackage{amssymb}
\usepackage{hyperref}


\usepackage[colorinlistoftodos]{todonotes}
\title{\textbf{Open Information Systems\\Final Report}}
\subtitle{2015-2016 Vrije Universiteit Brussel\\Responsible tutor : Christophe Debruyne - chrdebru@vub.ac.be}
\author{\textbf{Trackerspace}\\Matteo Marra - mmarra@vub.ac.be - 0525999 - 1M Comp Sci Soft\\Antoine Carpentier - antcarpe@vub.ac.be - 0529527 - 1M Comp Sci AI\\Titouan Christophe - tichrist@vub.ac.be - 0529190 - 1M Comp Sci Soft\\Bruno Rocha Pereira - brochape@vub.ac.be - 0529512 - 1M Comp Sci Soft}
\date{Exam Session: June 2016}
\begin{document}
\maketitle
\chapter{Introduction}
In this final report we will present our non-trivial demonstrator, \textit{Trackerspace} a web application that allows to navigate through running/biking tracks and exercise spots, that a user might use to record his workouts and look for a place where to do sport.
Our server provides a \textit{SPARQL} endpoint that can be used to query the data in our database and retrieve information according to four different ontologies, carefully linked between each other to provide a complete open information system.

We will begin describing the different ontologies we used and how we mapped to them, followed by the tools we used to develop and run our non-trivial demonstrator to finish with a description on what we achieved with our application.
\chapter{Ontologies}
\section{Overall ontology}
%blablabla on the ontology
The non-trivial demonstrator embraces the concepts of the class ontology, called \textit{overall ontology} in this document.
This ontology defines the concept of \texttt{Calorie}, defined as follows: ``A calorie is a unit of energy, it is gained by eating and lost during physical activity. A calorie equals 4.184 Joules".
\todo{insert how we mapped to it }

\section{Exercise ontology}
%blablabla on the ontology and how we mapped it
As for the overall ontology, the application follows the concepts defined in the \textit{exercise ontology} \todo{maybe add a link to the ontology?}, such as \texttt{Workout}, \texttt{Exercise}, \texttt{UserAccount} and \texttt{Person}. 
We encountered some differences between our application and how the ontology was defined, mainly because our application is totally different from the other applications following the same ontology. The main problem for us was that there is no concept like geographic position in the ontology, and our application is mostly based on that, since we model tracks. 

To map to the concept of \texttt{Equipment}, that is a property of an \texttt{Exercise}, we used the ID of the \texttt{Track}, since is the thing that makes more sense in the view of our application and since, choosing this solution, we wouldn't need to change too much the \textit{exercise ontology}. 
This showed us the need to create an ontology particular to our application that could map those concepts that were missing in the \textit{exercise ontology}, allowing us to relate also to the \textit{geosparql ontology} and create an end-point complete, that covers every functionality of our application.

We did not map to some concepts that, also together with the other groups in the developing process of the ontology, we reputed not necessary, such as \texttt{HearthRate} and \texttt{Repeats}.
\section{Geosparql ontology}
%blablabla on geosparql and how we mapped to it [problems we had go here?]
The OGC GeoSPARQL standard \footnote{http://www.opengeospatial.org/standards/geosparql} is a standard for representation and querying of geospatial data for the Semantic Web. It defines an ontology in \texttt{RDFS/OWL} and a \texttt{SPARQL} query interface.

It defines a class \texttt{geo:SpatialObject} and subclasses \texttt{geo:Feature} and \texttt{geo:Geometry}. It also defines a property of \texttt{geo:Feature}, called \texttt{geo:hasGeometry} that can link to a geometry.
Different types of geometries can be described, such as \texttt{Point}, \texttt{Polygon}, \texttt{Linestring} and many other.
It uses the \textit{Well Known Text (WKT)} format to serialize the different geometries.

\section{Our ontology}
%why we developed a new ontology
%the exercise ontology doesn't have any notation of position, we needed to create a new one to extend geosparql
As described in the previous chapters, our ontology covers the classes and concepts that were not described in the \textit{exercise ontology}, and allow us to link our geospatial data to the \textit{GeoSPARQL ontology}, in order to increase the potential of our non-trivial demonstrator and allow to do geospatial queries on our tracks from the \texttt{SPARQL} endpoint.

In this ontology we define a \texttt{Track} and an \texttt{ExerciseSpot} that are both subclasses of \texttt{geo:Feature}. This allows to inherit the property \texttt{geo:hasGeometry}, that will be used in the mapping to the ontology to finally map the geometry of the tracks and exercise spots to the \textit{geoSPARQL ontology}.

We also define many properties like \texttt{isAlongTrack}, \texttt{hasTrack}, \texttt{performed}, \texttt{hasTags} and \texttt{hasReview}, that complete the set of data we have stored in our application.

\chapter{Tools}
\section{PostgreSQL and db2triples}
As we mentioned in earlier reports, we use PostgreSQL and PostGis to prepare our data in relational tables. We then use db2triples \footnote{\url{https://github.com/chrdebru/db2triples}} to map our entities to RDF triples. Our mapping is written in the r2rml language \footnote{\url{https://www.w3.org/TR/r2rml/}}, using the Turtle syntax. The mapping file is called \texttt{r2rml.ttl}.
%brief on db2triples, problems we had mapping to the geosparql

\section{Parliament}
Finally, we use Parliament \footnote{\url{http://parliament.semwebcentral.org/}} as our triple store. Parliament provide an indexed triples database, a web based data explorer, and a SPARQL endpoint which notably supports the geographical functions, needed to perform geospatial querying. Our demonstrator uses Parliament's SPARQL endpoint.

\section{Some issues we encounterd with the tools}
\subsection{Conversion to WKT format}
We spent some time trying to convert the geometries stored in PostGIS to WKT litterals using r2rml functions in db2triples. We also tried to make our triples using the GeoTriples \footnote{\url{https://github.com/LinkedEOData/GeoTriples}} tool. After many unsuccesful attempts, we finally found an easier way to it, using PostGIS. Using the \texttt{ST\_AsText} function, we convert the data to the Well Known Text format in the database query, and db2triples only cpoies it as a string in our triples. Since our specific ontology defines tracks and exercices spots as \texttt{geo:Feature} subclass, all the reasoners that might use our dataset will use those objects as geometries.

\subsection{Coordinates order in geometries}
Our initial dataset, provided in a previous iteration, included geographical data where coordinates where given in the format \texttt{latitude, longitude}. However, as we did not pay attention to this aspect before, other GIS applications and libraries use the format \texttt{longitude, latitude}. Therefore, in our first trials to query our dataset in SPARQL and display it on a map, we encountered bugs, such as erronated track lengths and tracks not found in a certain bounding box. We then converted our dataset to the \texttt{longitude, latitude} format to fit the GIS ontology.

\subsection{Elevation}
Our dataset also include elevation of the tracks, which are therefore stored as \texttt{LINESTRING Z} GIS types. Our first trials with GeoSPARQL queries didn't return any results, and we found out that projecting tracks in 2 dimensions (using PostGIS \texttt{ST\_Force2D} function in the mapping query) fixed the problem. We didn't had much time to investigate wether we could do it in GeoSPARQL. Also, the javascript library we use to convert geometries from WKT to geojson in our demonstrator, Wellknown.js \footnote{\url{https://github.com/mapbox/wellknown}}, does not support 3D geometries.

\chapter{Non trivial demonstrator}
\section{The application}
Our demonstrator provides an interface with a map where users are able to see all tracks in the displayed area, as well as the exercises spots along these tracks. The users are also able to find tracks by activity type. Moreover, given a track, a user will be able to find related exercises: exercises spots along the track, crossing or nearby tracks, or similar tracks by activity type or tags.

These functionalities can be found on buttons placed on the upper left of the map as buttons. The first button displays all the tracks that fit in the current shown map. The second one queries all the tracks that have at least once been used for biking and displays all of them on the map. The third and last button displays all the exercise spots on the map.

Clicking on a track gives the user more queries to do. When a user selects a track, its URI is shown and 4 buttons appear. The user can then choose to either display all the intersecting tracks, the exercise spots related to it, all the exercises that involved that track as well as the tracks the share similar particularities(e.g. tags). 
\section{Queries}
%description on what we achieved
\chapter{Conclusions}

\end{document}