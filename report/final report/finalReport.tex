\documentclass[11pt,a4paper]{article}
\usepackage[utf8]{inputenc}
\usepackage[english]{babel}
\usepackage{amsmath}
\usepackage{amsfonts}
\usepackage{amssymb}
\usepackage[colorinlistoftodos]{todonotes}
\author{Group 10: \\ Matteo Marra, Antoine Carpentier, \\Titouan Christophe, Bruno Rocha Pereira}
\title{Open Information Systems \\ Final Report}
\date{Exam Session: June 2016}
\begin{document}
\maketitle
\section{Introduction}
In this final report we will present our non-trivial demonstrator, \textit{Trackerspace} a web application that allows to navigate through running/biking tracks and exercise spots, that a user might use to record his workouts and look for a place where to do sport.
Our server provides a \textit{SPARQL} endpoint that can be used to query the data in our database and retrieve information according to four different ontologies, carefully linked between each other to provide a complete open information system.

We will begin describing the different ontologies we used and how we mapped to them, followed by the tools we used to develop and run our non-trivial demonstrator to finish with a description on what we achieved with our application.
\section{Ontologies}
\subsection{Overall ontology}
%blablabla on the ontology
The non-trivial demonstrator embraces the concepts of the class ontology, called \textit{overall ontology} in this document.
This ontology defines the concept of \texttt{Calorie}, defined as follows: ``A calorie is a unit of energy, it is gained by eating and lost during physical activity. A calorie equals 4.184 Joules".
\todo{insert how we mapped to it }

\subsection{Exercise ontology}
%blablabla on the ontology and how we mapped it
As for the overall ontology, the application follows the concepts defined in the \textit{exercise ontology} \todo{maybe add a link to the ontology?}, such as \texttt{Workout}, \texttt{Exercise}, \texttt{UserAccount} and \texttt{Person}. 
We encountered some differences between our application and how the ontology was defined, mainly because our application is totally different from the other applications following the same ontology. The main problem for us was that there is no concept like geographic position in the ontology, and our application is mostly based on that, since we model tracks. 

To map to the concept of \texttt{Equipment}, that is a property of an \texttt{Exercise}, we used the ID of the \texttt{Track}, since is the thing that makes more sense in the view of our application and since, choosing this solution, we wouldn't need to change too much the \textit{exercise ontology}. 
This showed us the need to create an ontology particular to our application that could map those concepts that were missing in the \textit{exercise ontology}, allowing us to relate also to the \textit{geosparql ontology} and create an end-point complete, that covers every functionality of our application.

We did not map to some concepts that, also together with the other groups in the developing process of the ontology, we reputed not necessary, such as \texttt{HearthRate} and \texttt{Repeats}.
\subsection{Our ontology}
%why we developed a new ontology
%the exercise ontology doesn't have any notation of position, we needed to create a new one to extend geosparql

\subsection{Geosparql ontology}
%blablabla on geosparql and how we mapped to it [problems we had go here?]

\section{Tools}
\subsection{Parliament}
%brief on what is parliament and describe how and why we used it
\subsection{db2triples}
%brief on db2triples, problems we had mapping to the geosparql
\section{Non trivial demostrator}
\subsection{The application}
\subsection{Queries}
%description on what we achived
\section{Conclusions}

\end{document}