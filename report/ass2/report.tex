\documentclass[a4paper]{article}

\usepackage[english]{babel}
\usepackage[utf8x]{inputenc}
\usepackage{amsmath}
\usepackage{graphicx}
\usepackage[colorinlistoftodos]{todonotes}
\usepackage{url}

\title{Open Information Systems\\Assignement part 2 }
\author{Matteo Marra, Antoine Carpentier, Titouan Christophe, Bruno Rocha Pereira}

\begin{document}
\maketitle

\todo{read over this fcking shit}

\section{Introduction}

Our application will help users to share and rate their favorite locations and tracks for exercises of all sorts. Users will interact in three ways : 

\begin{enumerate}
    \item Record and share tracks (where they run, bike...)
    \item Tag those tracks with specific keywords for other users to discover them
    \item Record and share spots where they exercise along tracks
\end{enumerate}

We first describe the conceptual schema of our database. The following sections develop each of the three points above.

\section{Description of the database}

The \textbf{USER} entity describe a user. The user has a \textbf{records} attribute which relates to \textbf{RECORD} and a \textbf{exercise\_ratings} which relates to \textbf{EXERCISE\_RATING} (more on that later). They also have the \textbf{weight}, \textbf{height}, \textbf{birth\_year} and \textbf{male} attributes used to compute their physical shape and calories burn. The two latter attributes are used to determine the age and gender of the user.

The \textbf{EXERCISE\_SPOT} entity describes a spot along a track. Its \textbf{geometry} attribute is a list of spatial coordinates managed by the PostgreSQL plugin PostGIS (\url{http://postgis.net}). This data will be retrieved from the user's device in the GPX file format (\url{http://www.topografix.com/gpx.asp}). It has a \textbf{rating} attribute related to an \textbf{EXERCISE\_RATING}.

The \textbf{EXERCISE\_RATING} entity describes the rating that a \textbf{USER} gives to an \textbf{EXERCISE\_SPOT}.The \textbf{grade} attribute is a scale of 0 to 5 and the optional \textbf{review} is a short description/explaination of the rating.

The \textbf{TRACK} entity describes a track. Its \textbf{geometry} attribute is similar to the one of \textbf{EXERCISE\_SPOT}. Its \textbf{length} attribute is computed in the database during the creation of a track and stored.

The \textbf{TAG} entity describes a tag (i.e. a keyword) related to a track. The \textbf{TAGGED} entity is the link between a \textbf{TRACK} and a \textbf{TAG} because the relation between \textbf{TAG} and \textbf{TRACK} is many-to-many.

The \textbf{RECORD} entity describes the route of a user along a \textbf{TRACK}. It has the \textbf{start\_time} and \textbf{end\_time} self-explained attributes. The \textbf{activity\_type} attribute relates to the type of activity (running, walking...). The \textbf{calories} attribute is computed in the database when the user has finished based on the \textbf{activity\_type}, the \textbf{length} of the \textbf{track} and the \textbf{weight}, \textbf{height}, age and gender of the user. The \textbf{rating} attribute relates to a \textbf{RATING} working the same way as an \textbf{EXERCISE\_RATING}.

\section{Tracks}

When a user wants to record its use of a new or existing track, our application will track its movement and record the starting and ending time. The user will have to specify which activity type he is doing (e.g. running, biking...) and optionally rate their record (with a grade and a review). Our application will be able to determine if the user is using a new track, in which case, the application will record it automatically.
\todo{how to display tracks ? on a map ?}

The length of the track will be computed only once when the user saves and stored in the database for performance issues. It will indeed never change and should be stored even if it redundant.

\section{Tags}

Users will be able to tag tracks with keywords and search tracks based on these keywords. Each track will have none, one or several tags attached to it and one tag will be able to relate to several tracks. That way, users will find tracks related to their interests and methods of exercising.

\section{Exercise spots}

Along the tracks, users will be able to specify spots where they exercise (a specific location where they practice a sport or a part of a track that they like). They will be allowed to rate these spots with a grade and an optional review to permit other users to discover these spots.
When a user views a specific track, the spots will be displayed along their ratings.

\end{document}